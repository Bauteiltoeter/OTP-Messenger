\section{Projektbeschreibung}
\label{sec:projektbeschreibung}

Im Rahmen dieses Projektes soll eine Möglichkeit für jedermann geschaffen
werden, mit der in der Theorie absolut sicheren Verschlüsselungstechnologie
\glqq One-Time pad\grqq (OTP) verschlüsselt zu Kommunizieren. Dabei wird durch Hard-
und Softwaremaßnahmen auf der Endgeräteseite die Integrität der Nachrichten
gewährleistet. 

\section{Anforderungen}
\label{sec:requirements}

\subsection{Anforderungen die Hardware}
\label{sec:anforderungen-hardware}

\subsubsection{HW-REQ100: Einplatinen-Computer}
\label{sec:hwreq100}
Als Grundgerüst für die Bedieneinheit soll ein Einplatinen-Computer vom Typ
\glqq Raspberry Pi\grqq verwendet werden. Die Netzwerkbuchse auf dem Raspberry
PI wird hardwareseitig deaktiviert, so wird das unauthorisierte Eindringen über
das Netzwerk verhindert. 

\subsubsection{HW-REQ200: Erweiterungskarte für den RasPi}
\label{sec:hwreq200}
Auf einer eigens entwickelten Erweiterungskarte für
den Raspberry Pi werden drei Subsysteme untergebracht:
\begin{itemize}
	\item Der Zufallszahlengenerator. Als Vorlage hierfür dient das
		\hyperlink{http://www.jtxp.org/tech/xr232usb.htm}{XR232USB-Projekt}. Die Platine
		muss so gestaltet sein, dass der Zufallsgenerator mit aufgelöteten
		Weißblech-Gehäusen gegen Störeinstrahlung geschützt werden kann. 
	\item Die RS232-Verbindung. Die Erweiterungskarte nimmt Daten über die
serielle Schnittstelle entgegen und gibt sie wahlweise direkt über einen
dreipoligen Verbinder weiter, oder aber wandelt diese über einen
USB-RS232-Wandler in ein USB-Signal um. Dabei muss die Hardware so gestaltet
werden, das weder eine USB-Verbindung bei abgeschaltetem RasPi, noch eine nicht
mit Strm versorgte USB-Verbindung bei aktiviertem RasPi der Hardware schaden
zufügen können.
	\item USB-Hub. Um die fünf benötigen USB-Buchsen (Tastatur/Maus \ref{sec:hwreq400}, Zufallszahlengenerator, zwei SD-Kartenleser \ref{sec:hwreq300})
 zu verfügung zu stellen, muss ein Hub auf der Erweiterungskarte untergebracht werden.
\end{itemize}

\subsubsection{HW-REQ300: Zufallszahlenspeicher}
\label{sec:hwreq300}
Die vom Zufallszahlengerator erzeugten Zufallsdaten werden auf zwei SD-Karten
gespeichert. Beide sind mit USB an dem Raspberry Pi verbunden.

\subsubsection{HW-REQ400: Eingabemöglichkeiten}
\label{sec:hwreq400}
Die Bedienung des Geräts muss über Maus und Tastatur geschehen.

\subsubsection{HW-REQ500: Message relay}
\label{sec:hwreq500}
Als Message relay ins Internet dient ein standard-Bürocomputer.

\subsubsection{HW-REQ600: Server}
\label{sec:hwreq600}
Als message relay-server dient ein normaler Server mit rootzugang.

\subsubsection{HW-REQ700: Funkmodulator}
\label{sec:hwreq700}
Als Alternative zum Internet sollen zwei Teilnehmer auch direkt mit einer
Funkverbindung verschlüsselt kommunizieren können. Der Modulator muss direkt
über eine serielle Schnittstelle mit der Erweiterungskarte (\ref{sec:hqreq200}
verbunden werden können-

\subsection{Anforderungen an die Software}
\label{sec:anforderungen-software}

\subsubsection{SW-REQ100: Betriebssystem}
\label{sec:swreq100}
Auf dem unter \ref{sec:hwreq100} definiertem Einplatinen-Computer läuft eine
gehärtete Linux-Distribution. Sie ist auf ein Minimum abgespeckt um eine
möglichst geringe Angriffsfläche zu bieten.

\subsubsection{SW-REQ200: Bootvorgang}
\label{sec:swreq200}
Direkt nach dem Booten soll das Betriebssystem automatisch die Chatsoftware
starten. Ein Wechsel auf den Desktop des Betriebssystems sowie das Beenden der
Chatsoftware muss verhindert werden. 

\subsubsection{SW-REQ300: GUI}
\label{sec:swreq300}
Die grafische Oberfläche des Programms muss folgende Anforderungen erfüllen:
\begin{itemize}
\item Anzeige von Freunden
\item Anlegen von Freunden
\item Zufallszahlengeneration steuern und füllen der zwei SD-Karten mit den neuen
Zufallsdaten
\item Anzeigen der noch verfügbaren Entropie
\item Mit Freunden chatten 
\end{itemize}

\subsubsection{SW-REQ400: Zufallsgenerator}
\label{sec:swreq400}
Der Zufallsgenerator muss den USB-Zufallsgenerator ansprechen und die
Zufallsdaten auf SD-Karte speichern können. Dabei müssen die Zufallsdaten in
kleineren Blöcken gespeichert werden.

\subsubsection{SW-REQ500: Crypto-Modul}
\label{sec:swreq500}
Das eigentliche Cryptomodul beinhaltet folgende Funktionen, die von der GUI
aufgerufen werden können:
\begin{itemize}
\item Anlegen eines Accounts, festlegen einer eindeutigen User-ID
(\ref{sec:swreq600})
\item Verwalten von Freunden
\item Nachrichten verschlüsseln
\item Nachrichten entschlüsseln
\item Verwendete Zufallsdaten müssen auf der SD-Karte sofort mit Nullen
überschrieben werden 
\end{itemize}

\subsubsection{SW-REQ600: User-ID}
\label{sec:swreq600}
Für den Endbenutzer ist die User-ID eine Base36-kodierte 32-Bit Zahl. Beim ersten Start
der Software wird der Benutzer aufgefordert, an einem Computer mit
Internetzugang eine neue Benutzer-ID anzufordern und auf dem Endgerät
einzutippen. Der Master-Server hat eine Liste aller noch verfügbaren User-IDs,
beim Besuch der Webseite des Projekts wird bei jedem Laden eine neue Generiert
und angezeigt. 

 





