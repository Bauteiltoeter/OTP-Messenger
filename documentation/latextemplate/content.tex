\section{Wie diese Vorlage funktioniert}
\label{sec:vorlage-funktioniert}

Um ein neues PDF-Dokument im einheitlichen Stil zu erstellen, einfach diesen
Ordner mitsamt des Unterordners /latexmagic in ein neues Verzeichnis kopieren.

\subsection{template.tex}
\label{sec:template-tex}
Die Datei template.tex ist die Hauptdatei, in ihr wird die Überschrift, das
Datum sowie der Author bestimmt. Der Name dieser Datei bestimmt den Namen des
reesultierenden PDFs, sie sollte also bei einem neuen Dokument umbenannt werden.

\subsection{content.tex}
\label{sec:content-tex}
Diese Datei beinhaltet den eigentlichen Inhalt des PDFs. Sie darf nicht
unbenannt werden, an sonsten kann die Latexmagic sie nicht mehr finden. 
Natürlich können bei umfangreichen Dokumenten an dieser Stelle weitere
*.tex-Dateien eingebunden werden.

\subsection{Datei compileren}
\label{sec:datei-compilieren}
Zum Compilieren in eine PDF-Datei wird einfach das Kommando pdflatex
template.tex angewandt. Damit Inhalts- und Abbildungsverzeichnis erstellt
werden, muss der Befehl zwei mal hinter einander ausgeführt werden.

\subsection{Design-Änderungen}
\label{sec:design-changes}
Änderungen am Design sollten immer ins LaTeX-Template übernommen werden. Diese
Vorlage ist weit davon entfernt, gut zu sein. Vor allem in der package-Liste
befinden sich viele Pakete, die zur Zeit keine Verwendung finden. Dies liegt
daran, das ich selber diese Vorlage nur schnell aus anderen Projekten zusammen
gefriemelt habe und selber große Teile von anderen Leuten übernommen habe.
