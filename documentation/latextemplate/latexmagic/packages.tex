\documentclass[a4paper,oneside,12pt,halfparskip]{scrartcl} %fleqn for formula left align
%%%%%%%%%%%%%%%%%%%%%%%%%%%%%%%%%%%%%%%%%%%%%%%%%%%%%%%%%%%%%%%%%%%%%%%
%usepackages
\usepackage[utf8]{inputenc} %Kodierung
\usepackage[english, ngerman]{babel} %Sprache
\usepackage{geometry}
\geometry{a4paper, top=25mm, left=32mm, right=25mm, bottom=30mm, headsep=10mm, footskip=10mm} %Ränder
\usepackage[T1]{fontenc}
\usepackage{graphicx}	%includegraphic
\usepackage[dvipsnames, table]{xcolor} %For xcolor
\usepackage[final]{pdfpages}	%pdf include
\usepackage{amssymb} %math	
\usepackage{amsmath}
\usepackage{lipsum}
\usepackage{colortbl}
\usepackage[automark]{scrpage2} % for header
\usepackage[toc, page]{appendix} %appendix
\usepackage{longtable}
\usepackage{cite}  %for quotes
\usepackage{siunitx} %units
\usepackage{kpfonts} %font
\usepackage{tikz} %tikz
\usepackage{tikz-3dplot}
\usepackage[only-used=false]{acro} %Nomenklatur
\usepackage{hyperref} %pdf Eigenschaften
\usepackage{pgfplots} %plots
\usepackage[siunitx, european]{circuitikz} %si units
\usepackage{setspace} %big/medspace
\usepackage{multicol} %cells in tabular that have multiple rows
\usepackage{pifont}
\usepackage{lscape} %landscape mode
\usepackage{standalone} %include .tex standalones
\usepackage{setspace}
\usepackage{wrapfig} 
\usepackage{listings}
\usepackage{multirow}




%%%%%%%%%%%%%%%%%%%%%%%%%%%%%%%%%%%%%%%%%%%%%%%%%%%%%%%%%%%%%%%%%%%%%%%
%Header
\automark[subsection]{section} % subsection/section in header

%%%%%%%%%%%%%%%%%%%%%%%%%%%%%%%%%%%%%%%%%%%%%%%%%%%%%%%%%%%%%%%%%%%%%%%
%Counter
\newcounter{romanmark}

%%%%%%%%%%%%%%%%%%%%%%%%%%%%%%%%%%%%%%%%%%%%%%%%%%%%%%%%%%%%%%%%%%%%%%%
%Colors
\definecolor{ulgray}{gray}{.85}
\definecolor{plotc1}{gray}{0}
\definecolor{plotc2}{gray}{0}
\definecolor{plotc3}{gray}{0}
\definecolor{plotc4}{gray}{0}
\definecolor{plotc5}{gray}{0}

%%%%%%%%%%%%%%%%%%%%%%%%%%%%%%%%%%%%%%%%%%%%%%%%%%%%%%%%%%%%%%%%%%%%%%%
%%Tabular Settings
%%%\newcolumntype{g}{>{\columncolor{lightgray}}c}

%%%%%%%%%%%%%%%%%%%%%%%%%%%%%%%%%%%%%%%%%%%%%%%%%%%%%%%%%%%%%%%%%%%%%%%
%Fonts
\setkomafont{disposition}{\bfseries}

%%%%%%%%%%%%%%%%%%%%%%%%%%%%%%%%%%%%%%%%%%%%%%%%%%%%%%%%%%%%%%%%%%%%%%%
%PGFplots
\usepgfplotslibrary{units}
\newlength{\plotheight}
\newlength{\plotwidth}
\setlength{\plotheight}{5cm}
\setlength{\plotwidth}{13cm}
\newlength{\ulength}
\setlength{\ulength}{15cm}
\newlength{\imgwidth}
\setlength{\imgwidth}{10cm}
\pgfplotsset{compat=1.10}

%%%%%%%%%%%%%%%%%%%%%%%%%%%%%%%%%%%%%%%%%%%%%%%%%%%%%%%%%%%%%%%%%%%%%%%
%Tikz
\usetikzlibrary{shapes, arrows, backgrounds, patterns, intersections, automata}

%%%%%%%%%%%%%%%%%%%%%%%%%%%%%%%%%%%%%%%%%%%%%%%%%%%%%%%%%%%%%%%%%%%%%%%
%Formula
%\setlength{\mathindent}{25pt} %align formula left
\numberwithin{equation}{section} %formula index with section number
%%%%%%%%%%%%%%%%%%%%%%%%%%%%%%%%%%%%%%%%%%%%%%%%%%%%%%%%%%%%%%%%%%%%%%%
%Commands
\newcommand{\uparagraph}[1]			{\paragraph*{#1}\mbox{}\\}
\newcommand{\dwn}[1]				{_{\text {#1}}}
\newcommand{\tilec}[1]				{\tilde{\vec{#1}}}
\newcommand{\ucenter}[1]			{\begin{figure}[!ht] \begin{center} #1 \end{center} \end{figure}}
\newcommand{\uccenter}[2]			{\begin{figure}[!ht] \begin{center} #1  \caption{#2} \end{center} \end{figure}}
\newcommand{\uclcenter}[3]			{\begin{figure}[!ht] \begin{center} #1  \caption{#2} \label{#3} \end{center} \end{figure}}
\newcommand{\linebreakcell}[2][l]	{\begin{tabular}[#1]{@{}l@{}}#2\end{tabular}}
\newcommand{\xmark}					{\ding{55}}
\newcommand{\cmark}					{\ding{51}}
\newcommand{\utype}[1]				{\textbf{#1}}
\newcommand{\ukey}[1]				{\textit{#1}}
\newcommand{\uemph}[1]				{\emph{#1}}

\newcommand{\uwidearrowidth}		{.7}
\newcommand{\uvwidearrow}[4]		{\node at($(#2)!.8!(#3)$) (dummy){}; \draw[#1] ($(#2)-(\uwidearrowidth/2,0)$)--++(\uwidearrowidth,0) |- ($(dummy)+(\uwidearrowidth,0)$) -- (#3) -- ($(dummy)-(\uwidearrowidth,0)$) -| ($(#2)-(\uwidearrowidth/2,0)$);}% \node [minimum height = 0pt, minimum width = 0pt, fill=white, scale=.5, rotate= -90] at($(#2)!.5!(#3)$){#4}; }
\newcommand{\uvdwidearrow}[4]		{\node at($(#2)!.2!(#3)$) (dummy1){}; 
									 \node at($(#2)!.8!(#3)$) (dummy2){}; 
									 \draw[#1] (#2) -- ($(dummy1)-(\uwidearrowidth ,0)$)--++(\uwidearrowidth/2,0) |- ($(dummy2)-(\uwidearrowidth ,0)$) -- (#3)
									 --($(dummy2)+(\uwidearrowidth ,0)$) -| ($(dummy1)+(\uwidearrowidth/2 , 0)$) --++(\uwidearrowidth/2,0) -- (#2);}
									%\node [minimum height = 0pt, minimum width = 0pt, fill=white, scale=.5, rotate= -90] at($(#2)!.5!(#3)$){#4}; }

\newcommand{\imgcentered}[3]        {\begin{figure} \centering \includegraphics[width=0.5\textwidth]{#1} \caption{ #2 } \label{ #3 } \end{figure}}

\newcommand{\abs}[1]				{\ensuremath{\left\vert#1\right\vert}} 

\DeclareMathOperator{\atan}{atan}


%%%%%%%%%%%%%%%%%%%%%%%%%%%%%%%%%%%%%%%%%%%%%%%%%%%%%%%%%%%%%%%%%%%%%%%
%Paper info

\title{\worktitl}
\subtitle{\subtitl}
\author{\uauthor}
\date{\workdat}

\addto\captionsngerman{\renewcommand{\figurename}{Abb.}}


%%%%%%%%%%%%%%%%%%%%%%%%%%%%%%%%%%%%%%%%%%%%%%%%%%%%%%%%%%%%%%%%%%%%%%%
%PDF Eigenschaften
\hypersetup{
hidelinks,
pdftitle={\worktitl}
pdfauthor={\uauthor},
pdfsubject={\worktitl}
pdfkeywords={\worktitl}
}


